%%% Question 1 - - - - - - - - - - - - - - - - - - - - - - - - - - - - - -
\question This question is about functional programming as a programming paradigm.
\begin{parts}
\part Reduce all of the following Haskell expressions to normal forms. Your answers \emph{must} include all reduction steps. No marks are awarded for answers which just state the normal form.
\begin{subparts}
    \subpart[2] \haskellIn{([tail, drop 4, take 2] !! 1) "hope"} \droppoints
    \begin{solution} \emph{Comprehension/Application.} For all parts of this question: 2 marks for a full, complete trace. 1 mark if there is one \emph{minor} error. 0 marks otherwise.
\begin{verbatim}
   ([tail, drop 4, take 2] !! 1) "hope"
=> ([drop 4, take 2] !! 0) "hope"
=> drop 4 "hope"
=> drop 3 "ope"
=> drop 2 "pe"
=> drop 1 "e"
=> drop 0 ""
=> ""
\end{verbatim}
\end{solution}
    \subpart[2] \haskellIn{map (+) [1, 2, 3]} \droppoints
    \begin{solution} \emph{Comprehension/Application.} 
\begin{verbatim}
   map (+) [1, 2, 3]
=> ((+) 1) : map (+) [2, 3]
=> ((+) 1) : ((+) 2) : map (+) [3]
=> ((+) 1) : ((+) 2) : ((+) 3) : map (+) []
=> ((+) 1) : ((+) 2) : ((+) 3) : []
== [(+) 1, (+) 2, (+) 3]
\end{verbatim}
    \end{solution}
    \subpart[2] \haskellIn{if True && x then (\x -> x) (\x -> x) else y} \droppoints
    \begin{solution} \emph{Comprehension/Application.} 
\begin{verbatim}
   if True && x then (\x -> x) (\x -> x) else y
-- the only redex is in the true branch
=> if True && x then (\x -> x) else y
\end{verbatim}
    \end{solution}
    \subpart[2] \haskellIn{map ($ x) $ drop 1 [even, odd, not . even]}  \droppoints
    \begin{solution} \emph{Comprehension/Application.} 
\begin{verbatim}
   map ($ x) $ drop 1 [even, odd, not . even]
=> map ($ x) $ drop 0 [odd, not . even]
=> map ($ x) [odd, not . even]
=> ($ x) odd : map ($ x) [not . even]
=> ($ x) odd : ($ x) (not . even) : map ($ x) []
=> ($ x) odd : ($ x) (not . even) : []
== [($ x) odd, ($ x) (not . even)]
== [odd $ x, (not . even) $ x]
== [odd x, (not . even) x]
\end{verbatim}
    \end{solution}
    \subpart[2] \haskellIn{foldr (\x r -> (.) not x : r) [] [not, (&&) True]} \droppoints
    \begin{solution} \emph{Comprehension/Application.} 
\begin{verbatim}
   foldr (\x r -> (.) not x : r) [] [not, (&&) True]
=> (\x r -> (.) not x : r) not 
      (foldr (\x r -> (.) not x : r) [] [(&&) True])
=> (\r -> (.) not not : r) 
      (foldr (\x r -> (.) not x : r) [] [(&&) True])
=> (\r -> (.) not not : r) 
      ((\x r -> (.) not x : r) ((&&) True) 
      (foldr (\x r -> (.) not x : r) [] []))
=> (\r -> (.) not not : r) 
      ((\x r -> (.) not x : r) ((&&) True) [])
=> (\r -> (.) not not : r) 
     ((\r -> (.) not ((&&) True) : r) [])
=> (\r -> (.) not not : r) 
     ((.) not ((&&) True) : [])
=> ((.) not not : ((.) not ((&&) True) : []))
== [not . not, not . ((&&) True)]
\end{verbatim}
    \end{solution}
\end{subparts}
\part Consider the following definition of \haskellIn{concatMap}, which combines the behaviour of the \haskellIn{concat} and \haskellIn{map} functions into one function:
\begin{minted}{haskell}
concatMap :: (a -> [b]) -> [a] -> [b]
concatMap _ []     = []
concatMap f (x:xs) = f x ++ concatMap f xs
\end{minted}
\begin{subparts}
    \subpart[4] \haskellIn{concatMap} above is defined in terms of explicit recursion. Define an equivalent function \haskellIn{concatMap'} in terms of a single call to \haskellIn{foldr} with suitable arguments so that it is not explicitly recursive. \droppoints
    \begin{solution} \emph{Application.} 
\begin{minted}{haskell}
concatMap' :: (a -> [b]) -> [a] -> [b]
concatMap' f = foldr (\x r -> f x ++ r) []
\end{minted}
    \end{solution}
    \subpart[6] Prove that your definition of \haskellIn{concatMap'} is equivalent to the provided definition of \haskellIn{concatMap}:

    \vspace*{0.2cm}

    $\forall f~::~a \to \hslist{b},~\mathit{xs}~::~\hslist{a} \quad . \quad \mathit{concatMap}~f~\mathit{xs} = \mathit{concatMap'}~f~\mathit{xs}$ \droppoints
    \begin{solution} \emph{Application.} 2 marks for the base case and 4 for the inductive case.
\begin{verbatim}
  concatMap f []
= {applying concatMap}
  []
= {unapplying foldr}
  foldr (\x r -> f x ++ r) [] []
= {unapplying concatMap'} 
  concatMap' f []

  concatMap f (x:xs)
= {applying concatMap}
  f x ++ concatMap f xs 
= {induction hypothesis}
  f x ++ concatMap' f xs 
= {applying concatMap'}
  f x ++ foldr (\x r -> f x ++ r) [] xs 
= {unapplying foldr}
  foldr (\x r -> f x ++ r) [] (x:xs)
= {unapplying concatMap'}
  concatMap' f (x:xs)
\end{verbatim} 
    \end{solution}
    \vspace*{0.2cm}
    \subpart[5] Suppose that we now want a function which combines the behaviours of \haskellIn{mapM} and \haskellIn{concat}, \emph{i.e.} a function a following type:
    
    \vspace*{0.2cm}

\begin{minted}{haskell}
concatMapM :: Applicative f => (a -> f [b]) -> [a] -> f [b]
\end{minted}

    \vspace*{0.2cm}

    Implement this function in terms of a single call to \haskellIn{foldr} with suitable arguments. \droppoints
    \begin{solution} \emph{Application.}
\begin{minted}{haskell}
concatMapM 
  :: Applicative f => (a -> f [b]) -> [a] -> f [b]
concatMapM f 
  = foldr (\x r -> (++) <$> f x <*> r) (pure [])
\end{minted}
    \end{solution}
\end{subparts}
\end{parts}