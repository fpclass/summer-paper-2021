%-----------------------------------------------------------
% LaTeX template for University of Warwick exams
%-----------------------------------------------------------

\usepackage{fancyeq}

% configure the heading 
\ModuleCode{CS141}
\ModuleName{Functional Programming} % for aep, this should be all caps
\ExamPeriod{Summer 2021}  
\ExamCode{CS1410\_B}
\TimeAllowed{2 hours} % either 2 hours or 3 hours
\QuestionInstructions{There are \textbf{SIX} questions. Candidates should attempt \textbf{FOUR} questions.}
\OtherInstructions{Instructions specific to this module:
\begin{itemize}
    \item The questions are not in order of difficulty.
    \item Unless stated otherwise, you should assume that library functions are defined as shown in the module guide.
\end{itemize}
General exam instructions follow on the next page.}

% use this if your exam paper contains sections or comment it out otherwise
% \NoBreakAfterQuestions

\setminted[text]{fontsize=\small}
\setminted[haskell]{fontsize=\small}
\setminted[bash]{fontsize=\small}
\newcommand{\haskellTopIn}[1]{\mintinline[fontsize=\small]{text}{#1}}
\newcommand{\haskellIn}[1]{\mintinline[fontsize=\small,breaklines]{text}{#1}}
\newcommand{\bashIn}[1]{\mintinline[fontsize=\small]{bash}{#1}}

\usepackage[nomap]{FiraMono}

\usepackage{amsmath}
\usepackage{microtype}
\DisableLigatures[f]{encoding = *, family = tt* }

\begin{document}
    \MakeHeading
    
    \begin{questions}
        %%% Question 1 - - - - - - - - - - - - - - - - - - - - - - - - - - - - - -
\question This question is about functional programming as a programming paradigm.
\begin{parts}
\part Reduce all of the following Haskell expressions to normal forms. Your answers \emph{must} include all reduction steps. No marks are awarded for answers which just state the normal form.
\begin{subparts}
    \subpart[2] \haskellIn{([tail, drop 4, take 2] !! 1) "hope"} \droppoints
    \begin{solution} \emph{Comprehension/Application.} For all parts of this question: 2 marks for a full, complete trace. 1 mark if there is one \emph{minor} error. 0 marks otherwise.
\begin{verbatim}
   ([tail, drop 4, take 2] !! 1) "hope"
=> ([drop 4, take 2] !! 0) "hope"
=> drop 4 "hope"
=> drop 3 "ope"
=> drop 2 "pe"
=> drop 1 "e"
=> drop 0 ""
=> ""
\end{verbatim}
\end{solution}
    \subpart[2] \haskellIn{map (+) [1, 2, 3]} \droppoints
    \begin{solution} \emph{Comprehension/Application.} 
\begin{verbatim}
   map (+) [1, 2, 3]
=> ((+) 1) : map (+) [2, 3]
=> ((+) 1) : ((+) 2) : map (+) [3]
=> ((+) 1) : ((+) 2) : ((+) 3) : map (+) []
=> ((+) 1) : ((+) 2) : ((+) 3) : []
== [(+) 1, (+) 2, (+) 3]
\end{verbatim}
    \end{solution}
    \subpart[2] \haskellIn{if True && x then (\x -> x) (\x -> x) else y} \droppoints
    \begin{solution} \emph{Comprehension/Application.} 
\begin{verbatim}
   if True && x then (\x -> x) (\x -> x) else y
-- the only redex is in the true branch
=> if True && x then (\x -> x) else y
\end{verbatim}
    \end{solution}
    \subpart[2] \haskellIn{map ($ x) $ drop 1 [even, odd, not . even]}  \droppoints
    \begin{solution} \emph{Comprehension/Application.} 
\begin{verbatim}
   map ($ x) $ drop 1 [even, odd, not . even]
=> map ($ x) $ drop 0 [odd, not . even]
=> map ($ x) [odd, not . even]
=> ($ x) odd : map ($ x) [not . even]
=> ($ x) odd : ($ x) (not . even) : map ($ x) []
=> ($ x) odd : ($ x) (not . even) : []
== [($ x) odd, ($ x) (not . even)]
== [odd $ x, (not . even) $ x]
== [odd x, (not . even) x]
\end{verbatim}
    \end{solution}
    \subpart[2] \haskellIn{foldr (\x r -> (.) not x : r) [] [not, (&&) True]} \droppoints
    \begin{solution} \emph{Comprehension/Application.} 
\begin{verbatim}
   foldr (\x r -> (.) not x : r) [] [not, (&&) True]
=> (\x r -> (.) not x : r) not 
      (foldr (\x r -> (.) not x : r) [] [(&&) True])
=> (\r -> (.) not not : r) 
      (foldr (\x r -> (.) not x : r) [] [(&&) True])
=> (\r -> (.) not not : r) 
      ((\x r -> (.) not x : r) ((&&) True) 
      (foldr (\x r -> (.) not x : r) [] []))
=> (\r -> (.) not not : r) 
      ((\x r -> (.) not x : r) ((&&) True) [])
=> (\r -> (.) not not : r) 
     ((\r -> (.) not ((&&) True) : r) [])
=> (\r -> (.) not not : r) 
     ((.) not ((&&) True) : [])
=> ((.) not not : ((.) not ((&&) True) : []))
== [not . not, not . ((&&) True)]
\end{verbatim}
    \end{solution}
\end{subparts}
\part Consider the following definition of \haskellIn{concatMap}, which combines the behaviour of the \haskellIn{concat} and \haskellIn{map} functions into one function:
\begin{minted}{haskell}
concatMap :: (a -> [b]) -> [a] -> [b]
concatMap _ []     = []
concatMap f (x:xs) = f x ++ concatMap f xs
\end{minted}
\begin{subparts}
    \subpart[4] \haskellIn{concatMap} above is defined in terms of explicit recursion. Define an equivalent function \haskellIn{concatMap'} in terms of a single call to \haskellIn{foldr} with suitable arguments so that it is not explicitly recursive. \droppoints
    \begin{solution} \emph{Application.} 
\begin{minted}{haskell}
concatMap' :: (a -> [b]) -> [a] -> [b]
concatMap' f = foldr (\x r -> f x ++ r) []
\end{minted}
    \end{solution}
    \subpart[6] Prove that your definition of \haskellIn{concatMap'} is equivalent to the provided definition of \haskellIn{concatMap}:

    \vspace*{0.2cm}

    $\forall f~::~a \to \hslist{b},~\mathit{xs}~::~\hslist{a} \quad . \quad \mathit{concatMap}~f~\mathit{xs} = \mathit{concatMap'}~f~\mathit{xs}$ \droppoints
    \begin{solution} \emph{Application.} 2 marks for the base case and 4 for the inductive case.
\begin{verbatim}
  concatMap f []
= {applying concatMap}
  []
= {unapplying foldr}
  foldr (\x r -> f x ++ r) [] []
= {unapplying concatMap'} 
  concatMap' f []

  concatMap f (x:xs)
= {applying concatMap}
  f x ++ concatMap f xs 
= {induction hypothesis}
  f x ++ concatMap' f xs 
= {applying concatMap'}
  f x ++ foldr (\x r -> f x ++ r) [] xs 
= {unapplying foldr}
  foldr (\x r -> f x ++ r) [] (x:xs)
= {unapplying concatMap'}
  concatMap' f (x:xs)
\end{verbatim} 
    \end{solution}
    \vspace*{0.2cm}
    \subpart[5] Suppose that we now want a function which combines the behaviours of \haskellIn{mapM} and \haskellIn{concat}, \emph{i.e.} a function a following type:
    
    \vspace*{0.2cm}

\begin{minted}{haskell}
concatMapM :: Applicative f => (a -> f [b]) -> [a] -> f [b]
\end{minted}

    \vspace*{0.2cm}

    Implement this function in terms of a single call to \haskellIn{foldr} with suitable arguments. \droppoints
    \begin{solution} \emph{Application.}
\begin{minted}{haskell}
concatMapM 
  :: Applicative f => (a -> f [b]) -> [a] -> f [b]
concatMapM f 
  = foldr (\x r -> (++) <$> f x <*> r) (pure [])
\end{minted}
    \end{solution}
\end{subparts}
\end{parts}
        %%% Question 2 - - - - - - - - - - - - - - - - - - - - - - - - - - - - - -
\question This question is about recursive and higher-order functions. 
\begin{parts}

\part[3] Suppose that you are playing a card game. All cards have a strength, which we can represent as a value of type \haskellIn{Int}. A hand of cards can therefore be represented as a value of type \haskellIn{[Int]}. The strength of a hand is determined by the card with the greatest strength in it. Given a hand $A$ with some number of cards $n$ and some other number $k$ so that $1 \leq k \leq n$, we want to determine the sum of the strengths of all hands of size $k$ that can be assembled using only cards from $A$. To begin, define a function 
\vspace*{0.2cm}
\begin{minted}{haskell}
ofSize :: Int -> [a] -> [[a]]
\end{minted}
\vspace*{0.2cm}
which calculates all subsequences of a particular length. For example, calling \haskellIn{ofSize 3 [2, 6, 2, 8]} should evaluate to \haskellIn{[[2, 6, 2], [2, 6, 8], [2, 2, 8], [6, 2, 8]]}. \droppoints

\begin{solution}
\emph{Application.}
\begin{minted}{haskell}
ofSize :: Int -> [a] -> [[a]]
ofSize k xs = 
  [ys | ys <- subsequences xs, length ys == k]
\end{minted}
\end{solution}

\part[3] With the help of \haskellIn{ofSize}, define a function
\vspace*{0.2cm}
\begin{minted}{haskell}
solve :: Int -> [Int] -> Int
\end{minted}
\vspace*{0.2cm}
which finds a solution for the problem. For example, \haskellIn{solve 3 [2,6,2,8]} should evaluate to \haskellIn{30} since $8 + 8 + 8 + 6 = 30$. \droppoints 

\begin{solution}
\emph{Application.}
\begin{minted}{haskell}
solve :: Int -> [Int] -> Int
solve k = sum . map maximum . ofSize k
\end{minted}
\end{solution}

\part[10] Solving this problem by explicitly finding all hands of the desired size as you have done for \haskellIn{ofSize} is an easy solution for this problem, but may be inefficient since there are $2^n$ subsequences to consider for a hand of size $n$. A better solution would never generate the hands of size $k$ to begin with. Instead we can calculate how many times each element of the input hand determines the strength of the resulting, $k$-sized hands. This can be accomplished by noting that the strongest card in a given hand will appear as many times as possible in the $k$-sized hands, followed by the next strongest card, and so on. For example, if the given hand is \haskellIn{[9,8,6,2,1]} and $k = 4$, then \haskellIn{9} appears in 4 $k$-sized hands. This can be calculated as the binomial coefficient $\binom{4}{3}$ since there are four cards, other than \haskellIn{9}, to choose from and we need to choose three more cards to make up a hand of size $k = 4$. The next strongest card \haskellIn{8} appears in $\binom{3}{3}$ $k$-sized hands. As a reminder, the binomial coefficient is defined as:
\begin{displaymath}
\binom{n}{k} = \frac{n!}{k!(n-k)!}
\end{displaymath}
Define a function
\begin{minted}{haskell}
calculate :: Int -> [Int] -> Int
\end{minted}
which, given $k$ and the cards in $A$ in descending order as arguments, should return the solution for the problem \emph{without} explicitly generating any of the $k$-sized hands. You should define any helper functions you need that are not in the standard library. \droppoints

\begin{solution}
    \emph{Application. 10 marks for a full, correct solution. 8 marks for a solution with minor mistakes or missing binomial coefficient function. 6 marks with both errors. 4 marks for a serious attempt with major errors. 2 marks for some attempt. 0 marks for no attempt. }
\begin{minted}{haskell}
bicoeff :: Int -> Int -> Int
bicoeff n m = fac n `div` (fac m * fac (n - m))
    where fac x = product [1..x]

calculate :: Int -> [Int] -> Int
calculate k xs = go (len-1) (k-1) xs
    where len = length xs

          go _ _ [] = 0
          go n m (x:xs)
            | n == m = x
            | otherwise = bicoeff n m * x + go (n-1) m xs
\end{minted}
\end{solution}

\pagebreak
\part[4] Complete the definition of
\begin{minted}{haskell}
calculateAll :: Foldable f => f (Int, [Int]) -> Int
calculateAll = foldr ??? 0
\end{minted}
by replacing the \haskellIn{???} with a suitable argument so that, for example, evaluating \haskellIn{calculateAll [(3, [2,6,2,8]), (4, [9,8,6,2,1])]} results in \haskellIn{44}. I.e. it finds the problem instance in the given data structure that results in the greatest overall strength. \droppoints

\begin{solution}
    \emph{Application.}
\begin{minted}{haskell}
calculateAll :: Foldable f => f (Int, [Int]) -> Int
calculateAll = foldr (\(k,xs) r -> max r (solve k xs)) 0
\end{minted}
\end{solution}

\part[5] Write a Haskell program
\begin{minted}{haskell}
main :: IO ()
\end{minted}
which reads a problem specification from the standard input. An example run of the program is shown below where lines starting with \haskellIn{>} represent standard output and those starting with \haskellIn{<} input from the user.
\begin{minted}{text}
> Enter k:
< 3
> Enter size of A:
< 4
> Enter elements of A:
< 2
< 6
< 2
< 8
> The solution is 30.
\end{minted}
The \haskellIn{>} and \haskellIn{<} symbols should not be included in the actual program output/input. \droppoints

\begin{solution}
    \emph{Application.}
\begin{minted}{haskell}
program :: IO ()
program = do
    putStrLn "Enter k:"
    k <- read <$> getLine
    putStrLn "Enter size of A:"
    n <- read <$> getLine
    putStrLn "Enter elements of A:"
    xs <- replicateM n (read <$> getLine)
    putStrLn $ "The solution is " <> show (solve k xs) <> "."
\end{minted}
\end{solution}

\end{parts}
        
%%% Question 3 - - - - - - - - - - - - - - - - - - - - - - - - - - - - - -
\question This question is about user-defined types and type classes. You may \textbf{not} make use of GHC extensions such as \texttt{\small DeriveFunctor}.
\begin{parts}
    \part Consider the following data type:
\begin{minted}{haskell}
data Layout a = Element a  
              | Vertical (Layout a) (Layout a)
              | Horizontal (Layout a) (Layout a)
\end{minted}
    The intuition here is that a layout consists of elements which can be composed vertically or horizontally.
    \begin{subparts}
        \subpart[4] The \haskellIn{Layout} type is a functor. Define a suitable instance of the \haskellIn{Functor} type class for it. Your instance should obey the functor laws, but you do \emph{not} need to prove this. \droppoints 
        
        \begin{solution}
            \emph{Application.} 4 marks for the definition.
\begin{minted}{haskell}
instance Functor Layout where
    fmap f (Element a) = Element (f a)
    fmap f (Vertical x y) = 
        Vertical (fmap f x) (fmap f y)
    fmap f (Horizontal x y) = 
        Horizontal (fmap f x) (fmap f y) 
\end{minted}
        \end{solution}

        \subpart[4] The \haskellIn{Layout} type is foldable. Define a suitable instance of the \haskellIn{Foldable} type class for it. \droppoints

        \begin{solution}
            \emph{Application.} 4 marks for the definition.
\begin{minted}{haskell}
instance Foldable Layout where
    foldr f z (Element a) = f a z
    foldr f z (Vertical x y) = 
        foldr f (foldr f z y) x
    foldr f z (Horizontal x y) = 
        foldr f (foldr f z y) x 
\end{minted}
        \end{solution}
    \end{subparts}

    \part Consider the following definition: 
\begin{minted}{haskell}
example :: Layout (Sized Char)
example = Vertical 
    (Element (Sized (15,42) 'a'))
    (Horizontal 
      (Element (Sized (4,16) 'b')) 
      (Element (Sized (23,8) 'c')))
\end{minted}
    
    \begin{subparts}
        \subpart[2] Define a suitable data type \haskellIn{Sized} so that the above definition is valid. \droppoints
        
        \begin{solution}
            \emph{Application.}
\begin{minted}{haskell}
data Sized a = Sized (Int,Int) a
\end{minted}
        \end{solution}
    
        \subpart[4] Define a function 
        \vspace*{0.2cm}
        \begin{minted}{haskell}
width :: Layout (Sized a) -> Int
        \end{minted}
        \vspace*{0.2cm}
        which calculates the width of a layout. For example, \haskellIn{width example} should evaluate to \haskellIn{27}. The width of horizontally composed layouts is the sum of the widths of the sub-layouts and the width of vertically composed layout is the maximum width of the sub-layouts. \droppoints
        
        \begin{solution}
            \emph{Application.} 
\begin{minted}{haskell}
width :: Layout (Sized a) -> Int
width (Element (Sized (x,y) a)) = x 
width (Vertical a b) = max (width a) (width b)
width (Horizontal a b) = width a + width b
\end{minted}
        \end{solution}
    
        \subpart[4] Define a corresponding function
        \vspace*{0.2cm}
        \begin{minted}{haskell}
height :: Layout (Sized a) -> Int
        \end{minted}
        \vspace*{0.2cm}
        which calculates the height of a layout. For example, \haskellIn{height example} should evaluate to \haskellIn{58}. \droppoints
        
        \begin{solution}
            \emph{Application.} 
\begin{minted}{haskell}
height :: Layout (Sized a) -> Int
height (Element (Sized (x,y) a)) = y 
height (Horizontal a b) = max (height a) (height b)
height (Vertical a b) = height a + height b
\end{minted}
        \end{solution} 
    
        \subpart[7] Suppose that we wish to render a given layout to a surface that uses a coordinate system where the origin is the top left corner and that we need to know the top left coordinate of each element in a layout for this purpose. Define a function 
        \vspace*{0.2cm}
        \begin{minted}{haskell}
pos :: Layout (Sized a) -> Layout (Sized (Int,Int))
        \end{minted}
        \vspace*{0.2cm}
        which, for every element in the layout, calculates its absolute position and replaces its value with the calculated position. For example, \haskellIn{pos example} should evaluate to the following: \pagebreak
\begin{minted}{haskell}
Vertical (Element (Sized (15,42) (0,0)))
  (Horizontal 
    (Element (Sized (4,16) (0,42))) 
    (Element (Sized (23,8) (4,42))))
\end{minted}
        If you require any helper functions or type class instances in addition to those already defined in earlier parts of this question, you should define those as well. \droppoints

        \begin{solution}
            \emph{Application. 1 mark for correct handling of Element. 3 marks for Vertical. 3 marks for Horizontal. 0 marks for no attempt.}
\begin{minted}{haskell}
instance Functor Sized where
  fmap f (Sized dims x) = Sized dims (f x)

adjust :: (Functor f, Functor g)
       => (a -> b) -> f (g a) -> f (g b)
adjust f = fmap (fmap f)

pos :: Layout (Sized a) -> Layout (Sized (Int, Int))
-- if we have a single element, 
-- then it is at the origin
pos (Element (Sized dims x)) = 
  Element (Sized dims (0,0))
-- in a vertical layout, b is located below
-- a so that b's y position is a's height
pos (Vertical a b) = 
  Vertical a' (adjust (\(x,y) -> (x,y+height a')) b')
  where a' = pos a 
        b' = pos b
-- in a horizontal layout, b is located to the
-- right of a so that b's x position is a's width
pos (Horizontal a b) = 
  Horizontal a' (adjust (\(x,y) -> (x+width a',y)) b')
  where a' = pos a 
        b' = pos b
\end{minted}
        \end{solution}
    \end{subparts}
\end{parts}
        \allowdisplaybreaks
%%% Question 4 - - - - - - - - - - - - - - - - - - - - - - - - - - - - - -
\question This question is about equational reasoning. 
\begin{parts}
    \part[4] Suppose that the \haskellIn{length} function is defined as:
    \begin{minted}{haskell}
    length :: [a] -> Int 
    length []     = 0 
    length (x:xs) = 1 + length xs
    \end{minted}
    Consider the following well known property which states that the \haskellIn{length} of two combined lists is the same as the sum of the individual lists' lengths:
    \begin{center}
        \haskellIn{length (xs ++ ys) == length xs + length ys}
    \end{center}
    Prove this property. You may assume standard properties of arithmetic. \droppoints 
    
    \begin{solution}
        \emph{Application.} 1 mark for the base case, 3 marks for the inductive step (in particular, 1 mark for the associativity step)
\begin{verbatim}
  length ([] ++ ys) 
= { applying ++ }
  length ys 
= { left identity of + }
  0 + length ys
= { unapplying length }
  length [] + length ys

  length ((x:xs) ++ ys)
= { applying ++ }
  length (x : (xs ++ ys))
= { applying length }
  1 + length (xs++ys)
= { induction hypothesis }
  1 + (length xs + length ys)
= { associativity of + }
  (1 + length xs) + length ys
= { unapplying length }
  length (x:xs) + length ys
\end{verbatim}
    \end{solution}
    
    \part[10] For a list of length $n$, there are $2^n$ many subsequences. We can describe this property using Haskell functions as:
    \begin{center}
        \haskellIn{length (subsequences xs) = 2^(length xs)}
    \end{center}
    You may assume that \haskellIn{subsequences} is defined as follows:
    \begin{minted}{haskell}
    subsequences :: [a] -> [[a]]
    subsequences [] = [[]]
    subsequences (x:xs) = ys ++ map (x:) ys
      where ys = subsequences xs
    \end{minted}
    With the help of the previous property, standard properties of arithmetic, and any additional lemma(s) you need, which must also be proved, prove this property. \droppoints
    
    \begin{solution}
        \emph{Application.}
        We first need to prove a lemma (1 mark for the base case and 3 marks for the inductive step):
        \begin{center}
            \haskellIn{length xs == length (map f xs)}
        \end{center}
        The proof is by induction on \texttt{\small xs}:
\begin{verbatim}
  length [] 
= { unapplying map }
  length (map f [])

  length (x:xs)
= { applying length }
  1 + length xs 
= { induction hypothesis }
  1 + length (map f xs)
= { unapplying length }
  length (f x : map f xs)
= { unapplying map }
  length (map f (x:xs))
\end{verbatim}
        Using this lemma, the previously proved property about length and append, and standard arithmetic properties, we can then prove the property about subsequences (2 marks for the base case and 4 marks for the inductive step):
\begin{verbatim}
Proof by induction on xs.

Base case: length (subsequences []) = 2^(length [])
  length (subsequences [])
= { applying subsequences }
  length [[]]
= { applying length }
  1 
= { arithmetic }
  2^1
= { unapplying length }
  2^(length [])

Inductive step: 
    length (subsequences (x:xs)) = 2^(length (x:xs))
    
  length (subsequences (x:xs))
= { applying subsequences }
  length ( subsequences xs ++ 
           map (x:) (subsequences xs))
= { length property }
  length (subsequences xs) 
  + length (map (x:) (subsequences xs))
= { lemma }
  length (subsequences xs) + length (subsequences xs)
= { arithmetic }
  2 * length (subsequences xs)
= { induction hypothesis }
  2 * 2^(length xs)
= { arithmetic }
  2^(1+length xs)
= { unapplying length }
  2^(length (x:xs))
\end{verbatim}
    \end{solution}
    
    \part Functions form a semigroup if their co-domain is a semigroup and a monoid if their co-domain is a monoid:  
\begin{minted}{haskell}
instance Semigroup b => Semigroup (a -> b) where
    (<>) f g = \x -> f x <> g x

instance Monoid b => Monoid (a -> b) where 
    mempty = \x -> mempty 
\end{minted}
    Prove that \emph{all} semigroup and monoid laws hold for these instances of the \haskellIn{Semigroup} and \haskellIn{Monoid} type classes.
    \begin{subparts}
        \subpart[3] Right identity: \haskellIn{f <> mempty = f}  \droppoints
        
        \begin{solution}
            \emph{Application.}
\begin{verbatim}
  f <> mempty
= { applying <> }
  \x -> f x <> mempty x
= { applying mempty }
  \x -> f x <> mempty
= { right identity law for type b }
  \x -> f x
= { eta-conversion }
  f
\end{verbatim}
        \end{solution}
        
        \subpart[3] Left identity: \haskellIn{mempty <> f = f} \droppoints
        
        \begin{solution}
            \emph{Application.}
\begin{verbatim}
  mempty <> f
= { applying mempty }
  \x -> mempty x <> f x
= { applying mempty }
  \x -> mempty <> f x
= { left identity law for type b }
  \x -> f x
= { eta-conversion }
  f
\end{verbatim}
        \end{solution}
        
        \subpart[5] Associativity: \haskellIn{f <> (g <> h) = (f <> g) <> h}  \droppoints
        
        \begin{solution}
            \emph{Application.}
\begin{verbatim}
  f <> (g <> h)
= { applying <> }
  \x -> f x <> ((g <> h) x)
= { applying <> }
  \x -> f x <> ((\y -> g y <> h y) x)
= { beta-reduction }
  \x -> f x <> (g x <> h x)
= { associativity for type b }
  \x -> (f x <> g x) <> h x
= { beta-reduction }
  \x -> (\y -> f y <> g y) x <> h x
= { unapplying <> }
  \x -> (f <> g) x <> h x
= { unapplying <> }
  (f <> g) <> h
\end{verbatim}
        \end{solution}
    \end{subparts}
\end{parts}
        
%%% Question 5 - - - - - - - - - - - - - - - - - - - - - - - - - - - - - -
\question This question is about functors, applicative functors, and monads. Consider the following data type definition in Haskell. You may assume that functor, applicative functor, and monad laws have been proved for all \haskellIn{Functor} instances except your own. You may \textbf{not} make use of GHC extensions such as \texttt{\small DeriveFunctor}. % or maybe pairs?
\begin{minted}{haskell}
data UwU w u = MkUwU u (w u)
\end{minted}
\begin{parts} 
    \part[7] Define a suitable instance of the \haskellIn{Functor} type class for the \haskellIn{UwU} type, adding suitable super-class constraints if necessary, and prove that your instance obeys \emph{both} functor laws. \droppoints 
    
    \begin{solution}
    Instance (2 marks):
    \begin{verbatim}
    instance Functor w => Functor (UwU w) where 
      fmap f (MkUwU u w) = MkUwU (f u) (fmap f w)
    \end{verbatim}
    Proofs (2 marks for identity, 3 marks for fusion):
    \begin{verbatim}
    fmap id (MkUwU u w)
    = { applying fmap }
    MkUwU (id u) (fmap id w)
    = { applying id }
    MkUwU u (fmap id w)
    = { identity law for type w }
    MkUwU u (id w)
    = { applying id }
    MkUwU u w
    = { unapplying id }
    id (MkUwU u w)
    
    fmap (f.g) (MkUwU u w)
    = { applying fmap }
    MkUwU ((f.g) u) (fmap (f.g) w)
    = { fusion law for type w }
    MkUwU ((f.g) u) (fmap f (fmap g w))
    = { applying function composition }
    MkUwU (f (g u)) (fmap f (fmap g w))
    = { unapplying fmap }
    fmap f (MkUwU (g u) (fmap g w))
    = { unapplying fmap }
    fmap f (fmap g (MkUwU u w))
    = { unapplying function composition }
    (fmap f . fmap g) (MkUwU u w)
    \end{verbatim}
    \end{solution}
    
    \part[18] Define a suitable instance of the \haskellIn{Applicative} type class for the \haskellIn{UwU} type, adding suitable super-class constraints if necessary, and prove that your instance obeys \emph{all four} applicative functor laws. \droppoints
    \begin{solution}
    Instance (4 marks):
    \begin{verbatim}
instance Applicative w => Applicative (UwU w) where 
  pure x = MkUwU x (pure x) 
  (MkUwU f w0) <*> (MkUwU x w1) = 
    MkUwU (f x) (w0 <*> w1)
    \end{verbatim}
    Proofs (3 marks for Identity, 3 marks for Homomorphism, 4 for Interchange, 4 for Composition):
    \begin{verbatim}
    Identity: pure id <*> v = v 
    pure id <*> (MkUwU u w) 
    = { applying pure }
    MkUwU id (pure id) <*> (MkUwU u w) 
    = { applying <*> }
    MkUwU (id u) (pure id <*> w)
    = { identity law for type w }
    MkUwU (id u) w
    = { applying id }
    MkUwU u w
    
    Homomorphism: pure f <*> pure x = pure (f x)
    pure f <*> pure x 
    = { applying pure twice } 
    MkUwU f (pure f) <*> MkUwU x (pure x) 
    = { applying <*> }
    MkUwU (f x) (pure f <*> pure x)
    = { homomorphism law for type w }
    MkUwU (f x) (pure (f x))
    = { unapplying pure }
    pure (f x)
    
    Interchange: u <*> pure y = pure ($ y) <*> u 
    MkUwU u w <*> pure y 
    = { applying pure }
    MkUwU u w <*> MkUwu y (pure y)
    = { applying <*> }
    MkUwU (u y) (w <*> pure y)
    = { interchange law for w }
    MkUwU (u y) (pure ($ y) <*> w)
    = { unapplying $ }
    MkUwU (($ y) u) (pure ($ y) <*> w)
    = { unapplying <*> }
    MkUwU ($ y) (pure ($ y)) <*> MkUwU u w
    = { unapplying pure }
    pure ($ y) <*> MkUwU u w
    
Composition: 
  pure (.) <*> u <*> v <*> w = u <*> (v <*> w)

  pure (.) <*> MkUwU a x <*> 
  MkUwU b y <*> MkUwU c z
= { applying pure }
  MkUwU (.) (pure (.)) <*> MkUwU a x <*> 
  MkUwU b y <*> MkUwU c z
= { applying <*> }
  MkUwU ((.) a) (pure (.) <*> x) <*>
  MkUwU b y <*> MkUwU c z
= { applying <*> }
  MkUwU ((.) a b) (pure (.) <*> x <*> y) <*>
  MkUwU c z
= { applying <*> }
  MkUwU ((.) a b c) (pure (.) <*> x <*> y <*> z)
= { composition law for type w }
  MkUwU ((.) a b c) (x <*> (y <*> z))
= { applying function composition }
  MkUwU (a (b c)) (x <*> (y <*> z))
= { unapplying <*> }
  MkUwU a x <*> MkUwU (b c) (y <*> z)
= { unapplying <*> }
  MkUwU a x <*> (MkUwU b y <*> MkUwU c z)
    \end{verbatim}
    \end{solution}
\end{parts}
        %%% Question 6 - - - - - - - - - - - - - - - - - - - - - - - - - - - - - -
\question This question is about type-level programming. Haskell's type system is sufficiently expressive to allow us to implement an expression language in the types. For this question, you may assume that all of GHC's language extensions are available to you. You may also assume that natural numbers and type-level addition are defined as follows:
\begin{minted}{haskell}
data Nat = Zero | Succ Nat

type family Add (n :: Nat) (m :: Nat) :: Nat where
    Add 'Zero y = y
    Add ('Succ x) y = 'Succ ('Add x y)
\end{minted}
For convenience, you can use type-level numeric literals so that e.g. \haskellIn{'Succ 'Zero} can equally be written as \haskellIn{1}.

\begin{parts}
    \part[3] Define a closed type family
    \begin{minted}{haskell}
If :: Nat -> k -> k -> k
    \end{minted}
    which implements a type-level if expression. If the first argument, of kind \haskellIn{Nat}, is \haskellIn{'Zero} the third argument (the false branch) should be returned. Otherwise the second argument (the true branch) should be returned. \droppoints 

    \begin{solution}
        \emph{Application.}
        \begin{minted}{haskell}
type family If (n :: Nat) (t :: k) (f :: k) :: k where
    If 'Zero t f = f 
    If x t f = t
        \end{minted}
    \end{solution}

    \part[4] \label{eval} We define a data type to represent a simple expression language:
    \begin{minted}{haskell}
data Expr = Val Nat
          | Plus Expr Expr
          | Cond Expr Expr Expr
    \end{minted}
    Define a closed type family
    \begin{minted}{haskell}
Eval :: Expr -> Nat
    \end{minted}
    which evaluates a type of kind \haskellIn{Expr} to a corresponding \haskellIn{Nat} type so that, for example, \haskellIn{'Cond ('Val 1) ('Plus ('Val 2) ('Val 4)) ('Val 3)} evaluates to \haskellIn{6}. The semantics for \haskellIn{'Cond} are the same as for \haskellIn{If}. \droppoints 

    \begin{solution}
        \emph{Application.}
        \begin{minted}{haskell}
type family Eval (e :: Expr) :: Nat where
    Eval ('Val n) = n
    Eval ('Plus l r) = Add (Eval l) (Eval r)
    Eval ('Cond c t f) = If (Eval c) (Eval t) (Eval f)
        \end{minted}
    \end{solution}

    \part[10] For Part \ref{eval} we simply evaluated expressions recursively. A different approach to this is to simplify \emph{one sub-expression at a time}. Define a closed type family
    \begin{minted}{haskell}
Step :: Expr -> Expr
    \end{minted}
    which implements call-by-value, left-to-right semantics. For example, \haskellIn{Step ('Plus ('Plus ('Val 0) ('Val 1)) ('Plus ('Val 2) ('Val 3)))} \linebreak should result in \haskellIn{'Plus ('Val 1) ('Plus ('Val 2) ('Val 3))}. That is:
    \begin{itemize}
        \item For \haskellIn{'Plus}, we should first evaluate the first argument by one step. Only if that is already a normal form, then we should evaluate the second argument by one step. Only if both arguments are already normal forms should we perform addition.
        \item Likewise for \haskellIn{'Cond} and its first argument, although the true and false branches should not be evaluated strictly -- if the first argument to \haskellIn{'Cond} is a normal form, then the appropriate branch should be returned as is.
        \item \haskellIn{'Val} is already a normal form so that \haskellIn{Step ('Val 4)} leaves the expression unchanged as \haskellIn{'Val 4}. \droppoints
    \end{itemize} 

    \begin{solution}
        \emph{Application. 1 mark for the head. 1 mark for Val. 4 marks for Plus. 4 marks for Cond. }
        \begin{minted}{haskell}
type family Step (e :: Expr) :: Expr where
    Step (Val n) = Val n
    Step (Plus (Val x) (Val y)) = Val (Add x y)
    Step (Plus (Val x) e1) = Plus (Val x) (Step e1)
    Step (Plus e0 e1) = Plus (Step e0) e1
    Step (Cond (Val Zero) t f) = f
    Step (Cond (Val x) t f) = t
    Step (Cond e t f) = Cond (Step e) t f
        \end{minted}
    \end{solution}

    \pagebreak
    \part We now wish to extend our expression language with variables. We modify the \haskellIn{Expr} kind with a new constructor as follows:
    \begin{minted}{haskell}
data Expr = ... | Var Nat
    \end{minted}
    Variable names are represented by numeric indices.
    \begin{subparts}
        \subpart[4] Define a closed type family 
        \vspace*{0.2cm}
        \begin{minted}{haskell}
Index :: [k] -> Nat -> k
        \end{minted}
        \vspace*{0.2cm}
        which looks up the element in a type-level list at the specified index. For example, \haskellIn{Index '[1,2,3] 1} should evaluate to \haskellIn{2}. \droppoints 

        \begin{solution}
            \emph{Application.}
            \begin{minted}{haskell}
type family Index (xs :: [k]) (i :: Nat) :: k where
    Index (x ': xs) Zero = x
    Index (x ': xs) (Succ n) = Index xs n
            \end{minted}
        \end{solution}
        
        \subpart[4] Modify the \haskellIn{Step} type family to
        \vspace*{0.2cm}
        \begin{minted}{haskell}
Step :: Expr -> [Expr] -> Expr
        \end{minted}
        \vspace*{0.2cm}
        and add suitable case(s) for \haskellIn{'Var} so that e.g. \haskellIn{Step ('Var 1) '['Val 4, 'Var 0]} evaluates to \haskellIn{'Var 0}. \droppoints 

        \begin{solution}
            \emph{Application.}
            \begin{minted}{haskell}
type family Step (e :: Expr) (env :: [Expr]) :: Expr where
    Step (Val n) env = Val n
    Step (Var i) env = Index env i
    Step (Plus (Val x) (Val y)) env = Val (Add x y)
    Step (Plus (Val x) e1) env = Plus (Val x) (Step e1 env)
    Step (Plus e0 e1) env = Plus (Step e0 env) e1
    Step (Cond (Val Zero) t f) env = f
    Step (Cond (Val x) t f) env = t
    Step (Cond e t f) env = Cond (Step e env) t f
            \end{minted}
        \end{solution}
    \end{subparts}    
\end{parts}


    \end{questions}
\end{document}